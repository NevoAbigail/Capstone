\documentclass{article}

\usepackage{booktabs}
\usepackage{tabularx}
\usepackage{float}
\usepackage[margin=1in]{geometry}
\usepackage{xr}
\usepackage[raggedrightboxes]{ragged2e}

\title{Hazard Analysis for SmartLock\\\progname}

\author{\authname}
\date{\today}

\input{../Comments}
%% Common Parts

\newcommand{\progname}{4TB6 - Mechatronics Capstone} % PUT YOUR PROGRAM NAME HERE
\newcommand{\authname}{Team \#5, Locked \& Loaded
\\ Abi Nevo, nevoa
\\ Elsa Bassi, bassie
\\ Steffi Ralph, ralphs1
\\ Abdul Iqbal, iqbala18
\\ Stephen De Jong, dejons1
\\ Anthony Shenouda, shenoa2} % AUTHOR NAMES                  

\usepackage{hyperref}
    \hypersetup{colorlinks=true, linkcolor=blue, citecolor=blue, filecolor=blue,
                urlcolor=blue, unicode=false}
    \urlstyle{same}


\newcounter{canum} %Critical Assumptions number
\newcommand{\lthecanum}{CA\thecanum}
\newcommand{\caref}[1]{CA\ref{#1}}

\newcounter{srnum} %Safety Requirement number
\newcommand{\lthesrnum}{SR\thesrnum}
\newcommand{\srref}[1]{SR\ref{#1}}

\begin{document}
\maketitle
\thispagestyle{empty}

\newpage
\pagenumbering{roman}

\begin{table}[hp]
\caption{Revision History} \label{TblRevisionHistory}
\begin{tabularx}{\textwidth}{llX}
\toprule
\textbf{Date} & \textbf{Developer(s)} & \textbf{Change}\\
\midrule
14-10-22 & Elsa & Added FMEA\\
14-10-22 & Abi & Added Critical Assumptions \& Safety Reqs\\
14-10-22 & Steffi & Intro, Scope \& Purpose of Hazard Analysis \& System Boundaries and Components\\
17-10-22 & Abi & Revisions to Safety Requirements\\
19-10-22 & Abi & Added probability and severity ratings to FMEA \\
19-11-22 & Steffi & Updates for grammar, formatting and terminology\\
23-11-22 & Steffi & Updates for consistency across documentation\\
03-03-23 & Abi & Updating according to revised SRS\\
04-04-23 & Elsa & Updating for final documentation\\
\bottomrule
\end{tabularx}
\end{table}

\newpage


\tableofcontents

\listoftables

\listoffigures

\newpage

\pagenumbering{arabic}

\section{Introduction}
The purpose of the SmartLock project is to design and build a product that will provide bicycle users with a safer, easier, and more accessible way to secure their bike(s) through their smartphone. Additionally, it will provide users with a GPS feature to locate the lock in case of bike theft or misplacement.  It will consist of a physical lock that mounts to a bike and a smartphone application that will function as the user interface through which the lock can be disengaged wirelessly, as well as be located and informed of the lock's battery percentage. The project will provide an engineering solution using wireless communication, mechanical design, and smartphone application development. More broadly, it seeks to encourage members of society to pursue biking, in both a transportation and recreational capacity, improving the health of society’s citizens and its environment.  

This document aims to outline the hazards that may face the SmartLock.  A hazard is defined in this document to be anything that puts the efficacy of the SmartLock at risk of failure or places the user in danger.  Throughout this document all potential hazards will be outlined. Finally, using hazard analysis techniques,  these risks will be mitigated. 

\section{Scope and Purpose of Hazard Analysis}

This project's scope is to create a device that securely locks and unlocks a bike, where the locking mechanism can be disengaged via a smartphone app and doesn’t impede the rider's ability to use the bike, which could cause a safety issue.  It is crucial to understand both the requirements of what the project is supposed to accomplish and all the risks that may accompany those requirements – this is the purpose of the hazard analysis.  Furthermore, this document will aim to assess the system boundaries, critical assumptions and safety requirements to predict the potential hazards' effects in order to preemptively add precautions and mitigate risk. 


\section{Definitions}

\begin{minipage}{\textwidth}
\renewcommand*{\arraystretch}{1.5}
\begin{tabular}{| p{0.23\textwidth} | p{0.77\textwidth} |}
 \hline
 Term & Definition \\ 
 \hline
 Hazard & An action that puts the efficacy of the SmartLock at risk of failure or places the user in danger.\\ 
  \hline
 System Failure & System Failure is a condition when the locking functionality of the SmartLock malfunctions at any stage in its engagement, hold or disengagement such that the lock is no longer secure.\\ 
  \hline
 Risk & A risk indicates a potential safety concern to the user.\\ 
  \hline
 Error & An error indicates a problem with the software that relates to the engagement for the lock.\\ 
  \hline
 Conflicts & A conflict indicates that an action is trying to be executed in the wrong state, ie. trying to engage the lock when the mechanism is open.\\ 
 \hline
\end{tabular}
\end{minipage}\\

\section{System Boundaries and Components}
The system can be broken into the following components and has the following boundaries:
\subsection {Physical Components}
Our physical components are the aspects that will be on the bike itself.
\subsubsection{Locking Mechanism}
The locking mechanism will be the component that ensures the security of the bike.
\subsubsection{Opening/Closing Mechanism}
The opening/closing mechanism is the component that will both attach the bike to an external frame and ensure that the wheels will stay connected to the bike when it is left.
\subsubsection{Battery}
The battery will be used to turn on the electromagnet which allows for the disengagement of the locking mechanism.  It will also be used to power the Arduino (microcontroller) which will allow the smartphone app to communicate with the bike lock.
\subsection {Software Components}
The software components that we will be using are related to our smartphone app.
\subsubsection {App}
The app component itself will be used to communicate with the physical components, via the Arduino, to give the user information on the status of the lock and battery and to allow the user to disengage the lock. 
\subsubsection {Geotagging Location Services}
The location service component will be used to communicate to the app where the bike was located upon engaging the lock, for the purpose of remembering where your bike was left.
\subsection {Boundaries}
\subsubsection{Bike Size}
The boundary that we need to work with on the physical components is the standard sizes of bikes so that the lock can be mounted properly.
\subsubsection{Standard External Frames}
The other physical boundary that we need to work within is the standard size/location of external frames which provides us with measurements for the open/closing mechanism that we must abide by.
\subsubsection{Current Technology}
The software boundary that we must remain within is the bounds of current technology; this is a very feasible and large boundary to work within as we do not plan to use any complex software.

\section{Critical Assumptions}

\begin{itemize}

\item[CA\refstepcounter{canum}\thecanum\label{CA1}:] Assume operator is not tampering or purposefully damaging the product.
\item[CA\refstepcounter{canum}\thecanum\label{CA2}:] Assume weather is typical of Canada, thus the device shall not be operated in temperatures outside the range of -30 degrees Celsius to +40 degrees Celsius. It shall also not be operated under rainfall greater than 5 millimetres per hour or wind speeds greater than 10 kilometres per hour.
\item[CA\refstepcounter{canum}\thecanum\label{CA3}:] Assume operator's smartphone (including all integrated technologies, like GPS) is functioning properly.
\item[CA\refstepcounter{canum}\thecanum\label{CA4}:] Assume geotagging and Bluetooth signals are receivable and transmittable; operator is in a location that can be properly triangulated (i.e., operator is not underground, etc.). 
\item[CA\refstepcounter{canum}\thecanum\label{CA5}:] Assume operator's bicycle has standard frame and dimensions, and functions properly. A standard frame is 15'' to 21'' long and 13'' to 20'' high, with a 28.6mm tube diameter. 
\item[CA\refstepcounter{canum}\thecanum\label{CA6}:] Assume operator's smartphone has power/is charged. 

\end{itemize}


\section{Failure Modes and Effects Analysis}
Likelihood and Severity are rated on a 1-10 scale, with 10 being the most probable/severe. 
\begin{table}[H]
\caption{Failure Modes and Effects Analysis}
\tiny
\begin{tabular}{| p{0.07\textwidth} | p{0.09\textwidth}  | p{0.1\textwidth} | p{0.16\textwidth} | p{0.13\textwidth} | p{0.1\textwidth} | p{0.08\textwidth} | p{0.04\textwidth} | p{0.04\textwidth} |p{0.023\textwidth} | }
\hline
\textbf{Design Function} & \textbf{Failure Mode} & \textbf{Failure \newline Effects} & \textbf{Failure Causes} & \textbf{Detection} & \textbf{Recommended Actions} & \textbf{Design Controls} & \textbf{Safety Req.}  & \textbf{Likeli-hood} & \textbf{Sev-erity}\\ \hline

The \newline intended user \newline engages and \newline disengages the \newline locking \newline mechanism & Male and female \newline locking ends are not \newline secured \newline together; the structural integrity of the lock is compromised & Bike not \newline secured \newline (vulnerable to theft or loss) by the intended user, an unintended user (thief) or independent lock failure & 1. Faulty \newline electromagnetic coil \newline 2. Battery supply \newline disrupted by faulty wire \newline 3. The battery can no longer supply voltage \newline 4. Misshapen mechanical locking component \newline 5. Water, cold \newline temperature or dirt damage \newline 6. Improper use & Perform inspection of locking \newline mechanism \newline internals by \newline opening it up with simple tools. Signs of deformation and/or breaking due to torsional shear stress may be visible & Replace: \newline - faulty \newline electromagnetic coil \newline - faulty wires \newline - faulty battery \newline - misshapen mechanical locking component & Mechanism to manually disengage provided & \hyperref[SR1]{SR1}, %\hyperref[SR2]{SR2}, 
FR4 & 3 & 10  \\ \hline

Attaches bike to an external frame or bike rack & a) Lock does not fit around external frame  \newline b) Lock is broken along its body and cannot move as intended & Bike cannot be secured to an external frame (vulnerable to theft or loss) & The lock is: \newline - too short \newline - too rigid or not \newline flexible enough to fit \newline - broken: a piece of lock has become stuck, loose or fallen off \newline - too wide to fit through an external frame \newline - used improperly & 1. Attempt to fit the lock to an external frame \newline 2. Perform inspection of physical lock to detect any components compromising structural integrity or any signs of deformation or breaking due to bending stress & 1. Find a \newline different \newline external frame that fits the lock \newline 2. Repair lock with spare pieces, tightening loose pieces or lubricating moving parts & Lock will be designed with high flexibility & \hyperref[SR2]{SR2} & a) 4 b) 2 & a)8 b)10 \\ \hline

Transmits and \newline receives signal to engage/ disengage locking mechanism from the app to the lock & Locking mechanism fails to \newline engage or \newline disengage; lock remains in an \newline undesired state & 1. If fails to engage, bike not secured (vulnerable to theft or loss) \newline 2. If fails to disengage, the bike cannot be detached from the external frame & 1. App malfunction; unable to prepare or receive signal \newline 2. Wireless connection from SmartLock to smartphone disrupted by external force \newline 3. Communication protocol error \newline 4. Battery supply \newline disrupted by faulty wire \newline 5. The battery can no longer supply \newline voltage to the transmitting/receiving unit & Locking \newline mechanism is stuck in an undesired state after multiple attempts to engage or disengage & 1. Reboot app  \newline 2. Replace any faulty wires \newline 3. Replace faulty battery \newline 4. Manually move \newline smartphone and SmartLock such that they are in closer proximity to each other & Long-lasting battery installed & NFR9, NFR10 & 3 & 10 \\ \hline

Transmits, receives \& \newline displays status \newline information (engaged/ \newline disengaged, battery \newline percentage) from the lock to the app & Status \newline information not shown on the app or is inaccurate & Accurate \newline information not known; battery may be low or require replacement and/or bike may not be secured \newline (vulnerable to theft or loss) & 1. Internal app malfunction or high latency \newline 2. Status information not transmitted or received (see ‘Transmits and receives engagement /disengagement signal from the App’ above) \newline 3. Smartphone malfunction or battery depletion \newline 4. Faulty status sensors & 1. The app appears to be malfunctioning (not loading, the screen is frozen or information appears to be inaccurate or lagging)  \newline 2. Status information is inaccurate upon inspection of the actual status of lock internals & 1. Reboot Smartphone \newline 2. Reboot App \newline 3. Replace faulty status sensors \newline 4. Charge smartphone & Ability to manually check status information & \hyperref[SR1]{SR1} & 3 & 7\\ \hline

Withstands water from rainfall & Water \newline appears to have \newline permeated the \newline SmartLock & 1. Electronics damaged \newline 2. Locking mechanism damaged \newline 3. Mechanical components rusted & 1. Ineffective \newline waterproofing \newline(permeable sealing) of locking mechanism, \newline electronics and \newline mechanical components \newline 2. Improper use (in inclement weather more severe than average rainfall) & Perform inspection of locking \newline mechanism, \newline electronics and mechanical \newline components. \newline Corrosion, damaged components or water observed. & Replace \newline water-damaged components & 1. The system is well sealed against the \newline environment. \newline 2. Aside from \newline housing, the lock system is composed of materials which resist corrosion & \hyperref[SR3]{SR3} & 8 & 8 \\ \hline

‘Geotags’ location of bike and displays on app & Location information not shown on the app or is inaccurate & Accurate \newline location \newline information not known; the user may not be able to locate bike & 1. Smartphone geotag software malfunction (inaccurate location recorded) \newline 2. Internal app \newline malfunction  \newline 3. Smartphone battery depletion \newline 4. Location geocached somewhere with poor satellite triangulation capabilities or poor cellphone service \newline 5. Data sharing issue with smartphone geotag software & 1. The app appears to be \newline malfunctioning (not loading, the screen is frozen or lagging or information appears to be inaccurate).  \newline 2. Geocached \newline location is \newline inaccurate when compared to the actual location \newline 3. Smartphone indicates battery or data sharing issue & 1. Reboot GPS software app \newline 2. Reboot smartphone \newline 3. Reboot App \newline 4. Charge smartphone \newline 5. Move to a location with better service and satellite triangulation capabilities & None & FR2, FR7 & 7 & 6\\ \hline

Contains and \newline carries all physical lock \newline components on the bike when not in use & Some or all physical lock components cannot safely be mounted on the bike due to the absence of proper \newline storage that \newline accommodates all \newline components & 1. Components placed in \newline inappropriate storage \newline locations such that they dangle off the bike or asymmetrically weigh down the bike \newline 2. Components \newline aren't mounted to the bike& 1. Physical lock \newline component storage system lacks space for all components \newline 2. Broken or \newline malfunctioning physical lock component storage system \newline 3. Physical lock \newline components too large to be mounted safely on the bike & Physical lock \newline components cannot be stored safely on the bike & Repair and/or expand faulty storage system & Initial check to ensure mounting system and \newline corresponding components function as intended &  FR5 & 2 & 5\\ \hline
\end{tabular}
\end{table}



\section{Safety and Security Requirements}
\subsection{New Requirements - October 2022}
The following requirements must be added to the SRS document in the Non-Functional Requirements Category:
\begin{itemize}
\item[SR\refstepcounter{srnum}\thesrnum\label{SR1}:] Batteries and other internal components must be accessible and replaceable (see NFR10 in SRS). 
%\item[SR\refstepcounter{srnum}\thesrnum\label{SR2}:] The locking mechanism shall be able to disengage manually (e.g., with a key), in addition to remotely. %% THIS IS A STRETCH GOAL
\item[SR\refstepcounter{srnum}\thesrnum\label{SR2}:] The lock can be used for many different models of mountain, city, and road bikes. (see NFR12 in SRS).
\end{itemize}

\subsection{Existing Requirements}

The following requirements have already been included in the SRS document, and are restated here for convenience: \\

 \noindent FR4: Lock must only be engaged/disengaged by the intended user(s). \newline 
 FR5: The lock can be mounted to the bike's frame. \newline
 NFR9: The battery must last for greater than one month and/or sixty rides before needing to be replaced or charged. \\


\section{Roadmap}
The safety requirements that will be implemented in the scope of Mechatronics Capstone 4TB6 are SR1 and SR2. They are vital to the functionality, safety and security of the SmartLock and are reasonably achievable given the constraints of the course, project and team. 

Other requirements were identified as being important for a high-quality project, but the team has decided it is not feasible to implement them within the time frame or resources of the project, and are therefore out of our scope. These requirements are:

\begin{itemize}
\item[SR\refstepcounter{srnum}\thesrnum\label{SR3}:] Product shall be made from anti-corrosive materials. %%SHOULD this be added as a stretch goal
\item[SR\refstepcounter{srnum}\thesrnum\label{SR4}:] The lock must be waterproofed to withstand normal rainfall of 5 millimetres per hour, which is typical of Canadian weather. 
\item[SR\refstepcounter{srnum}\thesrnum\label{SR5}:] The lock must be waterproofed to withstand normal splashing while riding, which is estimated at a distributed value of 1 millimetre over 10 minutes of riding.
\end{itemize}

 

\end{document}