\documentclass[12pt, titlepage]{article}

\usepackage{booktabs}
\usepackage{tabularx}
\usepackage{hyperref}
\usepackage{caption}
\hypersetup{
    colorlinks,
    citecolor=blue,
    filecolor=black,
    linkcolor=red,
    urlcolor=blue
}
\usepackage[round]{natbib}

%% Comments
\usepackage{color}
\newif\ifcomments\commentstrue %displays comments
%\newif\ifcomments\commentsfalse %so that comments do not display
\ifcomments
\newcommand{\authornote}[3]{\textcolor{#1}{[#3 ---#2]}}
\newcommand{\todo}[1]{\textcolor{red}{[TODO: #1]}}
\else
\newcommand{\authornote}[3]{}
\newcommand{\todo}[1]{}
\fi
\newcommand{\wss}[1]{\authornote{blue}{SS}{#1}} 
\newcommand{\plt}[1]{\authornote{magenta}{TPLT}{#1}} %For explanation of the template
\newcommand{\an}[1]{\authornote{cyan}{Author}{#1}}
%% Common Parts
\newcommand{\progname}{4TB6 - Mechatronics Capstone} % PUT YOUR PROGRAM NAME HERE
\newcommand{\authname}{Team \#5, Locked \& Loaded
\\ Abi Nevo, nevoa
\\ Elsa Bassi, bassie
\\ Steffi Ralph, ralphs1
\\ Abdul Iqbal, iqbala18
\\ Stephen De Jong, dejons1
\\ Anthony Shenouda, shenoa2} % AUTHOR NAMES                  

\usepackage{hyperref}
    \hypersetup{colorlinks=true, linkcolor=blue, citecolor=blue, filecolor=blue,
                urlcolor=blue, unicode=false}
    \urlstyle{same}


\newcounter{testnum} %Test Number
\newcommand{\dthetestnum}{TN\thetestnum}

\begin{document}

\title{Project Title: System Verification and Validation Plan for \progname{ Smart Bike Lock}} 
\author{Abi Nevo\\Elsa Bassi\\Steffi Ralph\\Abdul Iqbal\\Stephen De Jong\\Anthony Shenouda}
\date{\today}
	
\maketitle

\pagenumbering{roman}

\section{Revision History}

\begin{tabularx}{\textwidth}{p{2cm}p{2cm}p{2cm}X}
\toprule {\bf Date} & {\bf Version} & {\bf Name} & {\bf Notes}\\
\midrule
20-10-22 & 1.0 & Steffi & Section 2, 3 \& 4\\
01-11-22 & 1.1 & Stephen, Elsa, Abi, \& Anthony & Section 5\\
02-11-22 & 1.2 & Abdul & Section 4\\
\bottomrule
\end{tabularx}

\newpage

\tableofcontents

\listoftables


\newpage

\section{Symbols, Abbreviations and Acronyms}

Refer to Section 1 of the \href{https://github.com/NevoAbigail/Capstone/blob/main/docs/SRS/SRS.pdf}{SRS} documentation for a full reference section on units, symbols and abbreviations/acronyms.

\newpage

\pagenumbering{arabic}

This document will outline the verification and validation plans we have created for the purpose of proving that the SmartLock device is a successful product. The tests mentioned in this document will validate our product against the various requirements which we have outlined in the SRS document. At the completion of the plan in this document, we will have the knowledge required to form an iterative design process which improves into a successful device at its completion. 

\section{General Information}

\subsection{Summary}
\-\
The smart lock can be broken down into two main components, the software (ie. the smartphone application) and the hardware (ie. the physical lock).

The smartphone application, SmartLock, will be a basic UI that our users can interact with to send a signal to unlock/lock the locking mechanism, check in on the battery of the device, and remember where they left their bike when they locked it.

The physical locking device will be used to secure the wheels to the bike frame, which can also be connected to an external mount.

These components will be tested through various different methods, to determine if they meet all the requirements outlined in the \href{https://github.com/NevoAbigail/Capstone/blob/main/docs/SRS/SRS.pdf}{SRS}.

\subsection{Objectives}
\-\
The objective that our project hopes to accomplish is to create a simple-to-use smartphone app that allows users to lock the most important features of their bike with confidence and to allow them to find their bike with ease.

\subsection{Relevant Documentation}
\-\
For more information on the project breakdown, planning or delivery refer to the following documentation:
 \href{https://github.com/NevoAbigail/Capstone/blob/main/docs/SRS/SRS.pdf}{SRS},
 \href{https://github.com/NevoAbigail/Capstone/blob/main/docs/HazardAnalysis/HazardAnalysis.pdf}{HA}, MG, and MIS.
 

\section{Plan}
\-\
In this section, various aspects of the verification plan will be illustrated explaining the process from concept to implementation. Beginning with an outline of the Validation team and their roles in guiding the process, this section will cover the verification of our SRS documentation as well as how we plan to design, implement and test our problem and eventually prototype our solution. It also covers any automation tools used as well as a software validation plan.

\-\

  
~\newpage

\subsection{Verification and Validation Team}
\-\

 \begin{minipage}{\textwidth}
 \captionof{table}{Verification and Validation Team Table}
\renewcommand*{\arraystretch}{1.5}
\begin{tabular}{| p{0.30\textwidth} | p{0.70\textwidth} |}
 \hline
 VnV Team Members & Role \\ 
  \hline
 Group 5 Members & Responsible for quality control on their topic of expertise, keeping the other group members aware of any changes or progress and reflecting that knowledge in the documentation\\ 
 \hline
Gr 5: Abi Nevo & Wireless Communication Expert \\ 
  \hline
 Gr 5: Elsa Bassi & Microcontroller Expert \\ 
  \hline
 Gr 5: Steffi Ralph & Documentation Expert/ Faux Marker \\ 
  \hline
 Gr 5: Abdul Iqbal & Lock Mechanism Expert \\ 
  \hline
 Gr 5: Stephen De Jong & Lock Frame Design Expert \\ 
  \hline
  Gr 5: Anthony
  ~\newline Shenouda & Software Expert \\ 
  \hline
   Dr. Sirouspour & From our supervisor, Dr. Sirouspour, we are looking for technical feedback on components that we might not have considered or fully grasped as well as feedback on any design, materials or documentation considerations\\
  \hline
    Classmates & Peer review from the perspective of someone who knows what is required of the documentation \\
  \hline
   Nicholas Annable & Provide specific feedback and grades on the work we have completed so that we can take the appropriate action moving further into the project\\
  \hline
  Dr. Smith & Provide information and guidance on the project goals/deliverables and documentation requirements \\
  \hline

\end{tabular}
\end{minipage}\\

\subsection{SRS Verification Plan}
\-\
Our plan for verification of the SRS includes the following steps:
\begin{enumerate}
\item Keep the Document Alive
\subitem By keeping the document alive we will be able to continually update any requirements/constraints/objectives that adapt as our project comes to life.  This will ensure that all the documentation is accurate. 
\item Peer Review
\subitem We will use the peer review to verify our work and to look for inconsistencies or aspects that we didn't consider.
\item TA Review
\subitem The TA review will be used to validate what we have done and what needs to be reworked as the project progresses.
\item SRS Checklist
\subitem With every update of the SRS document and progression of the SmartLock we will continue to reference the  \href{https://github.com/NevoAbigail/Capstone/blob/main/docs/Checklists/SRS-Checklist.pdf}{SRS Checklist} to ensure that we continue to meet all the requirements and produce complete documentation.
\end{enumerate}

\subsection{Design Verification Plan}
\-\
Our plan for verification of the Design Plan includes the following steps:
\begin{enumerate}
\item Review with Supervisor (Dr. Sirouspour)
\subitem As a novice design team, we will use the meetings with Dr. Sirouspour to review, primarily the hardware components of, our design and understand if we are using the right technical pieces and if our approach can lead to success. 
\item Proof of Concept Demo
\subitem The proof of concept demo will be a crucial part of our validation plan.  With the intent to be able to demonstrate both a portion of our physical component and a preliminary software UI, the presentation and feedback will provide us with an opportunity to validate our objectives and verify the next steps and problem areas.
\item Testing
\subitem Testing, which will be discussed in detail below, will be used to understand if our requirements can be met given the scope of this project and help us to determine what needs to change for our final product.
\item MIS \& MG Checklists
\subitem The \href{https://github.com/NevoAbigail/Capstone/blob/main/docs/Checklists/MIS-Checklist.pdf}{MIS Checklist} and the
 \href{https://github.com/NevoAbigail/Capstone/blob/main/docs/Checklists/MG-Checklist.pdf}{MG Checklist} will be used to review the tests that are created and check that they follow the guidelines so that we can properly test the work product. Tests will be reviewed and updated as the project progresses.
\end{enumerate}

\-\

\subsection{Implementation Verification Plan}

Our plan to verify the implementation is outlined below:
\begin{enumerate}
\item Functional Requirements Verification

\subitem The verification of our functional requirements will be done by performing the system tests discussed in \nameref{Section 5.1}. The success rate of these system tests will help the team to confirm and validate these requirements and therefore will provide the team with good metrics on the core functionality of the product.

~\newpage

\item Non-Functional Requirements Verification

\subitem The verification of our non-functional requirements will be done by performing the system tests discussed in \nameref{Section 5.2}. The success rate of these system tests will help the team to confirm and validate these requirements and therefore will provide the team with good metrics on the overall quality and class of the product. This will help us in verifying other aspects of the product including the App features, durability and ease of use.

\item Code Implementation Verification

\subitem In order to verify the implementation of our code-base, we will make use of both code walk-throughs and inspection by fellow peers as well as by performing unit tests on our system. Having a peer review done would not only illuminate any shortcomings in our implementation but also help us in identifying more efficient techniques within our code-base. By performing unit tests, which are discussed in \nameref{Section 6}, we can verify different modules within our system to see if each is performing as expected and is robust. This will help us maintain best practices while implementing our code-base.

\end{enumerate}

\subsection{Automated Testing and Verification Tools}
The tools used for automated testing and verifying coding standards are outlined below:

\begin{enumerate}

\item SwiftLint

\subitem SwiftLint will be used in order to enforce common Swift style and naming conventions.

\item CodeFactor

\subitem CodeFactor will be used for automated code analysis on our repo on Github.

\item Github

\subitem Github will be used to host our codebase including all staging branches and the production branch. All changes will need to be approved and reviewed before merging to production.

\item Git

\subitem Git will be used for version control, resolving merge conflicts and interfacing with different branches.

\end{enumerate}

\subsection{Software Validation Plan}

\wss{If there is any external data that can be used for validation, you should
  point to it here.  If there are no plans for validation, you should state that
  here.}

\section{System Test Description}
	
\subsection{Tests for Functional Requirements}
\label{Section 5.1}

The tests in this section will be used to confirm and validate our functional requirements from the SRS document. Completing these tests will prove the functionality of our product

\subsubsection{Area of Testing: User Input Related}

\begin{enumerate}

\item{test-id1: FR1: LockEngage input must engage the lock on the bike. \\} 

Control: Automatic 

Initial State: Bike lock is disengaged 

Input/ Condition: Signal from App to engage the lock 

Output/ Results: Pass/fail if lock is successfully engaged  

How test will be performed: The bike lock will be disengaged. The user will use the App to engage the lock. If successful, lock becomes engaged. 

\item{test-id2: FR2: LockDisengage input must disengage the lock on the bike. \\} 

Control: Automatic 

Initial State: Bike lock is engaged 

Input/ Condition: Signal from App to disengage the lock 

Output/ Results: Pass/fail if lock is successfully disengaged  

How test will be performed: The bike lock will be engaged. The user will use the App to disengage the lock. If successful, lock becomes disengaged. 

\item{test-id3: FR3: When lock is engaged, change lock status to engaged. \\} 

Control: Automatic 

Initial State: Bike lock is engaged 

Input/ Condition: Signal from microcontroller to App stating lock is engaged 

Output/ Results: Pass/fail if App correctly displays lock is engaged 

How test will be performed: The bike lock will be engaged.  

 \item{test-id4: FR4: When lock is disengaged, change lock status to disengaged. \\} 

Control: Automatic 

Initial State: Bike lock is disengaged 

Input/ Condition: Signal from microcontroller to App stating lock is disengaged 

Output/ Results: Pass/fail if App correctly displays lock is disengaged 

How test will be performed: The bike lock will be disengaged. 

\item{test-id5: FR5: Moving locking bar closed must move OpenClosedStatus to Closed. \\} 

Control: Automatic 

Initial State: Locking bar is open 

Input/ Condition: Locking bar is moved closed 

Output/ Results: Pass/fail if App correctly displays OpenClosedStatus to Closed 

How test will be performed: The locking bar will be moved closed and the App will be checked if the OpenClosedstatus to Closed. 

\item{test-id6: FR6: Moving locking bar open must move OpenClosedStatus to Open. \\} 

Control: Automatic 

Initial State: Locking bar is closed 

Input/ Condition: Locking bar is pushed open 

Output/ Results: Pass/fail if App correctly displays OpenClosedStatus to Open 

How test will be performed: The locking bar will be pushed open and the App will be checked if the OpenClosedStatus to Open. 
 
\item{test-id7: FR7: Location (coordinates) of user’s phone must be able to be saved in the smartphone application as UserPosition. \\} 

Control: Automatic 

Initial State: No location  

Input/ Condition: User engages the Bike lock using the App 

Output/ Results: Pass/fail if App places a geotag within 10 metres of the Users current location 

How test will be performed: The Bike lock will be engaged on the App. The App will be monitored to ensure the geotag is stored and correctly placed on the App.  

\item{test-id8: FR8: Effective Bike Lock: The lock is sturdy and cannot be manually opened by the average human once engaged. \\}

Control: Manual

Initial State: Lock is engaged, and bike is locked.

Input/Condition: A prying, pulling, kicking, etc. force between 200-400N.

Output/Results: A pass/fail as well as a score from 1-4 for the following cases; a fail if the lock disengages and breaks, a fail if the lock disengages, a pass if the lock stays engaged but breaks, and a pass if the lock system can stay engaged with out breaking.

How test will be performed: A test group of 2-3+ adults, completing 50+ trials, each giving the required input forces.

\item{test-id9: FR9: Lock must only be engaged/disengaged by the intended user(s). \\}

Control: Manual

Initial State: Lock is engaged, and bike is locked.

Input/Condition: Iphone Bluetooth signal.

Output/Results: A fail if they are able to connect to the smart bike lock and disengage it and a pass if they can’t connect and disengage the lock.

How test will be performed: A test group will attempt to connect via Bluetooth from their smartphones.

\end{enumerate}

\subsubsection{Area of Testing: Bike Input Related}

\begin{enumerate}

\item{test-id10: FR10: The lock can be mounted to the bike’s frame. \\}

Control: Manual

Initial State: Lock is in disengaged state and not mounted to a bike.

Input/Condition: Users force to mount the SmartLock.

Output/Results: A pass/fail for each bike if the lock is/isn’t able to mount in to the bike as intended. A score from 0-the number of bikes tested will also be given.

How test will be performed: Users will attempt to mount the SmartLock onto 1-3 bikes each from 3-5 categories of bikes including (adult standard/hybrid, road, mountain, kids standard/hybrid).

\end{enumerate}

\subsubsection{Area of Testing: Output Related}

\begin{enumerate}

\item{test-id11: FR11: Battery percentage must be shown on the phone app. \\}
Control: Manual 

Initial State: Phone on with App downloaded and open 

Input/Condition: observation of app 

Output/Results: user must be able to view the battery percentage of the lock 

How the test will be performed: apply input condition, observe expected output

\item{test-id12: FR12: Location (coordinates) of bike must be shown on the app as BikePosition. \\}
Control: Manual 

Initial State: Phone on with App downloaded and open 

Input/Condition: observation of app 

Output/Results: user must be able to view the saved coordinates 

How the test will be performed: apply input condition, observe expected output

\item{test-id13: FR13: Battery must output enough power to engage the lock. \\}
Control: Manual 

Initial State: Locking mechanism disengaged, electro magnet has ability to get power from battery (functional circuit) 

Input/Condition: supply power to electromagnet 

Output/Results: lock mechanism is engaged 

How the test will be performed: apply input condition, observe expected output

\end{enumerate}

\subsection{Tests for Nonfunctional Requirements}
\label{Section 5.2}

This subset of tests will be used to validate the nonfunctional requirements of our product. Completing these tests will prove various aspects of our products needs. these aspects include smart phone app features, physical design attributes, accuracy, and the usability of our product.

\subsubsection{Area of Testing: Smart Phone}

\begin{enumerate}

\item{test-id14: NFR1: Can be used by people who speak any native language. \\}

Control: Manual

Initial State: Lock is engaged and phone is on with App downloaded and open.

Input/Condition: Non-English-speaking test user has no prior knowledge of how to use the System.

Output/Results: The lock is successfully engaged and the bike is securely locked by test user.

How test will be performed: User attempts to disengage the mechanism through the App, attach bike to external frame and re-engage mechanism.
					
\item{test-id15: NFR2: Can reasonably be used without requiring an instruction manual. \\}

Control: Manual

Initial State: Lock is engaged and phone is on with App downloaded and open.

Input/Condition: A group of test users with no prior knowledge of how to use the System. 

Output/Results: The lock is successfully engaged, and the bike is securely locked by all test users.

How test will be performed: Users are told to disengage the mechanism through the App, attach the bike to an external frame and re-engage the mechanism without any instruction. The time required for full usage per test is measured using a stopwatch and must below ten minutes. Learning period is subsequently calculated and plotted. 

\item{test-id16: NFR3: App storage under 50 megabytes. A small mobile app should not take significant space on the user’s phone.  \\}

Control: Static 

Initial State: App development IDE is open and up to date.

Output/Results: App storage is less than 50 megabytes.

How test will be performed: App storage displayed on the IDE is kept below the upper threshold of 50 megabytes. 

\end{enumerate}

\subsubsection{Area of Testing: Physical Lock}
\begin{enumerate}

\item{test-id17: NFR4: The design must be visually appealing. \\}
Control: Manual 

Initial State: Locking mechanism assembled 

Input/Condition: Survey users on their opinions of the visual appeal of the device; see \nameref{Survey}.

Output/Results: Visual appeal is rated 7 or higher (on a scale of 1-10) 

How the test will be performed: Apply input condition, survey 50 users and observe expected output.

\item{test-id18: NFR5: The lock must not impede normal bike functions. \\}
Control: Manual 

Initial State: Locking mechanism assembled and mounted on bike frame 

Input/Condition: Normal bike operation (ride forward, turn right, turn left, ride downhill, ride uphill) 

Output/Results: Device does not impede bicycle functioning 

How the test will be performed: apply input condition, observe expected output

\item{test-id19: NFR6: The lock must be waterproofed to withstand normal rainfall. And NFR7:  The lock must be waterproofed to withstand normal splashing while riding. \\}

Stage 1: Empty Housing (no electronics inside) 

Control: Manual 

Initial State: Husing assembled.

Input/Condition: Simulate rain: Hold housing under shower for 5 minutes, slowly rotating.

Output/Results: Receives a pass if the inside of housing is completely dry.

How the test will be performed: apply input condition, observe expected output

Stage 2: Full Mechanism 

Control: Manual 

Initial State: Locking mechanism, electronics, and housing assembled, microcontroller turned on.

Input/Condition: Simulate rain: hold housing under shower for 5 minutes, slowly rotating.

Output/Results: Receives a pass if the device remains functional and operational.

How the test will be performed: apply input condition, observe expected output

\item{test-id20: NFR8: Accuracy of bike lock status must be above 95\%. \\}
Control: Manual 

Initial State: Locking mechanism, electronics, and housing assembled operational and within range.

Input/Condition: Engage/disengage lock 100 times. 

Output/Results: Ensure engage/disengage status matches current physical state.

How the test will be performed: apply input condition, observe expected output
\end{enumerate}

\subsubsection{Area of Testing: Usability}

\begin{enumerate}

\item{test-id21: NFR11: iPhone app locking must be quicker to use than a typical keyed/combo bike lock.  \\}

Control: Manual 

Initial State: The Smart lock is in the transport state with the test user standing beside the bike at a bike rack.

Input/Condition: User engaging and disengaging the SmartLock.

Output/Results: Scores from 1-10 will be given for weighted time from fasted to slowest.

How test will be performed: Users will be timed on locking and unlocking. As a group of 3 people will, remove lock from the transport state, and engage the lock as intended. To lock they will disengage the lock as intended and convert to transport state. This will be done with 3-5 types of locks including, SmartLock, keyed, and combination.

\item{test-id22: NFR12: Opening and closing lock must require similar force to a typical keyed/combo.  \\}

Control: Manual.

Initial State: The Smart Lock will be in transport state with the test user standing beside the bike at a bike rack.

Input/Condition: Users Engaging and Disengaging the SmartLock.

Output/Results: Scores from 1-5 will be given from most to least amount of relative force required.

How test will be performed: Users will measure force while locking and unlocking. As a group of 3 people will, remove lock from the transport state, and engage the lock as intended. To lock they will disengage the lock as intended and convert to transport state. This will be done with 3-5 types of locks including, SmartLock, keyed, and combination.

\item{test-id23: NFR13: Battery must last for greater than 1 month and/or 60 rides before needing to be replaced or charged.  \\}

Control: Manual 

Initial State: The Smart Lock is fully charged.

Input/Condition: Users Engaging, Locating, and Disengaging the SmartLock.

Output/Results: The amount of time and quantity of lock/unlocks of the Smart Lock. A pass if it meets the required number.

How test will be performed:  A group of users will take turns to bike to a new bike lock, lock the bike, mark the bikes location in the app, unlock, and then repeat. This will be done until the battery dies.

\item{test-id24: NFR14: Batteries must be accessible to replace or chargeable. \\}

Control: Manual.

Initial State: The Smart Lock is in the Transport state.

Input/Condition: Users will charge/replace batteries as intended.

Output/Results: Pass if the batteries can be charged/replaced as intended.

How test will be performed: Users will charge/replace batteries as intended.

\item{test-id25: NFR15: The lock must be easily mounted on the bike frame. It does not require special tools, (i.e., those not found in a typical toolbox, such as power tools), to be installed and does not take more than twenty minutes to install.  \\}

Control: Manual.

Initial State: The System is ready to be installed.  

Input/Condition: A group of test users with an average understanding of how to use typical tools. They have access to those typical tools. 

Output/Results: The System is successfully mounted on the bike frame without special tools for each test user. 

How test will be performed: Each user attempts to install the System, following the procedure outlined in the instructions. This procedure will be developed following the completion of the first stage of prototyping. 

\item{test-id26: NFR16: The lock can be used for many different models of mountain, city, and road bikes.  \\}

Control: Manual 

Initial State: The System is ready to be installed. 

Input/Condition: A group of test users with an average understanding of how to use typical tools. They have access to those typical tools. Three different models of bike are tested; one for each type (mountain, city and road). 

Output/Results: The System can be mounted on all three bike models successfully by each test user. 

How the test will be performed: Three users attempt to install the System, following the procedure outlined in the instructions.  

\end{enumerate}

\subsubsection{Area of Testing: Accuracy}

\begin{enumerate}

\item{test-id20: NFR10: Battery percentage must be calculated accurately within 10\%. \\}

Control: Manual

Initial State: Full device assembled, battery percentage calculated is 1\%. 

Input/Condition: Engage/disengage lock until battery dies.

Output/Results: Number of lock engages possible with 1\% battery matches our specification for number of lock engages possible with 100\% battery (multiply the measured \# of lock engages possible with 1\% battery by 100).

How the test will be performed: Apply input condition and observe expected output

\end{enumerate}

\subsection{Traceability Between Test Cases and Requirements}

\begin{minipage}{\textwidth}
\footnotesize
\captionof{table}{Traceability Table}
\renewcommand*{\arraystretch}{1.5}
\begin{tabular}{| p{0.2\textwidth} | p{0.4\textwidth} | p{0.45\textwidth} |}
 \hline
 Test Case & \href{https://github.com/NevoAbigail/Capstone/blob/main/docs/SRS/SRS.pdf}{Functional Requirement(s)} & \href{https://github.com/NevoAbigail/Capstone/blob/main/docs/SRS/SRS.pdf}{Non-Functional Requirement(s)} \\ 
 \hline
 test-id1 & FR1 &  \\ 
  \hline
 test-id2 & FR2 &  \\
  \hline
 test-id3 & FR3 &  \\  
  \hline
 test-id4 & FR4 &  \\ 
  \hline
 test-id5 & FR5 &  \\ 
  \hline
 test-id6 & FR6 &  \\ 
  \hline
 test-id7 & FR7 &  \\ 
  \hline
 test-id8 & FR8 &  \\ 
  \hline
 test-id9 & FR9 &  \\ 
  \hline
 test-id10 & FR10 &  \\ 
  \hline
 test-id11 & FR11 & \\ 
  \hline
 test-id12 & FR12 &  \\ 
  \hline
 test-id13 & FR13 &  \\ 
 \hline
 test-id14 & & NFR1 \\
 \hline
  test-id15 & & NFR2 \\
 \hline
  test-id16 & & NFR3 \\
 \hline
  test-id17 & & NFR4 \\
 \hline
  test-id18 & & NFR5 \\
 \hline
  test-id19 & & NFR6, NFR7 \\
 \hline
  test-id20 & & NFR8 \\
 \hline
  test-id21 & & NFR11 \\
 \hline
  test-id22 & & NFR12 \\
 \hline
  test-id23 & & NFR13 \\
 \hline
  test-id24 & & NFR14 \\
 \hline
  test-id25 & & NFR15 \\
 \hline
  test-id26 & & NFR16 \\
 \hline
 test-id27 & & NFR10 \\
 \hline
 \end{tabular}
\end{minipage}\\

Note: NFR9 regarding GPS location accuracy has been changed to a constraint due to its dependence on the GPS capabilities of the user's smartphone. These capabilities are not associated with the System, therefore it is not included in the Traceability Table.

\section{Unit Test Description}
\label{Section 6}

\subsection{Unit Testing Scope}

Not applicable.

\subsection{Tests for Functional Requirements}

Not applicable.

\subsection{Tests for Nonfunctional Requirements}

Not applicable.

\subsection{Traceability Between Test Cases and Modules}

Not applicable.
				
\bibliographystyle{plainnat}

\bibliography{../../refs/References}

\newpage

\section{Appendix}

\subsection{Usability Survey Questions}

\begin{enumerate}

\item{testid-10: Survey Question}
\label{Survey}

Rate this device's visual appeal on a scale of 1-10, with 10 being most visually appealing, 1 being least visually appealing.

\end{enumerate}


\newpage{}
\section*{Appendix --- Reflection}

The information in this section will be used to evaluate the team members on the
graduate attribute of Lifelong Learning.  Please answer the following questions:

\begin{enumerate}
  \item What knowledge and skills will the team collectively need to acquire to
  successfully complete the verification and validation of your project?
  Examples of possible knowledge and skills include dynamic testing knowledge,
  static testing knowledge, specific tool usage etc.  You should look to
  identify at least one item for each team member.



  \item For each of the knowledge areas and skills identified in the previous
  question, what are at least two approaches to acquiring the knowledge or
  mastering the skill?  Of the identified approaches, which will each team
  member pursue, and why did they make this choice?
\end{enumerate}

\end{document}
