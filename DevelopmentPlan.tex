\documentclass{article}
\usepackage{graphicx}
\usepackage{booktabs}
\usepackage{tabularx}

\title{Development Plan\\\progname}

\author{\authname}

\date{}

\input{Comments.tex}
%% Common Parts

\newcommand{\progname}{4TB6 - Mechatronics Capstone} % PUT YOUR PROGRAM NAME HERE
\newcommand{\authname}{Team \#5, Locked \& Loaded
\\ Abi Nevo, nevoa
\\ Elsa Bassi, bassie
\\ Steffi Ralph, ralphs1
\\ Abdul Iqbal, iqbala18
\\ Stephen De Jong, dejons1
\\ Anthony Shenouda, shenoa2} % AUTHOR NAMES                  

\usepackage{hyperref}
    \hypersetup{colorlinks=true, linkcolor=blue, citecolor=blue, filecolor=blue,
                urlcolor=blue, unicode=false}
    \urlstyle{same}


\begin{document}

\begin{table}[hp]
\caption{Revision History} \label{TblRevisionHistory}
\begin{tabularx}{\textwidth}{llX}
\toprule
\textbf{Date} & \textbf{Developer(s)} & \textbf{Change}\\
\midrule
Sept 25 & Abi & Drafted Team Meeting, Communication Plan and Team Member Roles\\
Sept 26 & Elsa & Drafted Workflow Process\\
\bottomrule

\end{tabularx}
\end{table}

\newpage

\maketitle

\wss{Put your introductory blurb here.}

\section{Team Meeting Plan}

Our team will have a meeting weekly on Mondays at 10:30 AM in Thode library.  Should we decide on Monday that we need another meeting that week, we will have a meeting on Thursdays at 2:30 PM, also in Thode.  If more meeting time is still necessary, our group will arrange for a meeting on Friday or over the weekend. 

Meetings with our supervisor, Dr. Sirouspour, will occur biweekly (once every two weeks) on Wednesdays from 11 AM to 12 PM virtually via Teams. 

An agenda will be created and committed to our Git repo prior to meetings.  This responsibility will be assigned to a different team member each week, on a rotation, however, all team members may suggest agenda topics.  The agenda will include suggested meeting topics, who will present/lead each topic, and an estimated time of discussion. Meeting minutes will also be recorded in the same document as the agenda, by the same team member, and be uploaded to our Git repo after the meeting.  This team member will also be responsible for chairing the meeting. 

\section{Team Communication Plan}

Our primary form of fast, short communication will be through our text group chat. We will use Github to post issues and host documentation for deliverables.  In addition to Github, we will also use Trello for project management and task tracking (Kanban).  Lastly, we will use Teams to host virtual meetings.

\section{Team Member Roles}

\begin{table}[h]
\caption{Team Member Roles} \label{TblTeamMemberRoles}
\begin{tabularx}{\textwidth}{llX}
\toprule
\textbf{Role} & \textbf{Primary Lead} & \textbf{Support}\\
\midrule
Microcontroller Design & Elsa & Abi\\
Documentation/Latex/"Faux Marker"/Git & Steffi & Elsa\\
Lock Frame Design (CAD) & Stephen & Steffi\\
Software (App Development) & Anthony & Abdul\\
Software (Wireless Communication) & Abi & Anthony\\
Lock Mechanism Design (CAD) & Abdul & Stephen\\
\bottomrule
\end{tabularx}
\end{table}

Each member has been designated as a lead for the various features/topics of our project, however, each member should have some knowledge of every aspect.  Team members should be prepared to be flexible and give support to whichever aspect/feature has current need. 

\section{Workflow Plan}

\subsection{Code Development}

Team members are expected to use the team repository on Github for code development. The master branch will be used as the current working copy of the code. To develop code, they will fork the main branches to create sub-branches that can then be merged back to the main branch according to the rules outlined below. 

\subsection{Rules for Merging}

Merging will take place under the following branch protection rules: 

\begin{enumerate}

\item The main branch can only be merged into it and not committed to directly. 
\item Merging pull requests is disallowed when tests fail. 
\item Pull requests are required before merging such that one other group member must approve the code. 
\item Status checks and actions must pass before merging. 
\item Branches must be up to date before merging to reduce merge conflicts. 

\end{enumerate} 

\subsection{Commits}

Commits should take place as often as possible, preferably for every 50 lines of code. Commit messages must be specific, concise and descriptive. 

\subsection{Workflow Process}

Our workflow process will be as follows.

\begin{enumerate}
\item Create a detailed and structured plan for the software development. 
\item Pull any new changes from the master branch. 
\item Create a new branch to develop on. 
\item Implement any independent modules. 
\item Perform unit testing on the independent modules. 
\item Push the changes made to the new branch.  
\item Implement any dependent modules .
\item Perform unit testing on the dependent modules.  
\item Merge all changes to the main branch after a pull request is approved by another group member.  
\end{enumerate}

\subsection{Issue Tracking}

Github Issues will be used to track bugs and for accountability purposes. An issue will be created when a developer is unsure about their path forward or if a bug is detected in the code. Issues will be assigned to specific team members explicitly. When an issue is detected and the team member is not able to resolve it themselves, the team lead for that issue will review and advise. Issues can also be raised if a team member foresees a conflict or problem with merging as well as for general tasks that may not be being accomplished according to the deadlines specified in the Milestones. 
\bigskip
They will be categorized according to the following labels.

\begin{itemize}

\item Bug: Code that is not working properly.
\item Documentation: Improvements or additions to documentation.
\item Help Wanted: Extra attention is needed.
\item Wontfix: This will not be worked on.
\item CAD: A task related to CAD in this location.
\item Wireless Communications: A task related to CAD in this location.
\item App: A task related to an application in this location.
\item Microcontroller Design: A task related to a microcontroller in this location.
\item Stretch Goal: A task related to a stretch goal in this location. 
\item Template: A task related to a template.
\end{itemize}

\subsection{Milestones}

Milestones will be created for each deliverable to keep team members accountable. This will be added continuously as necessary. 

\section{Proof of Concept Demonstration Plan}

What is the main risk, or risks, for the success of your project?  What will you
demonstrate during your proof of concept demonstration to convince yourself that
you will be able to overcome this risk?

\section{Technology}

\begin{itemize}
\item Specific programming language
\item Specific linter tool (if appropriate)
\item Specific unit testing framework
\item Investigation of code coverage measuring tools
\item Specific plans for Continuous Integration (CI), or an explanation that CI
  is not being done
\item Specific performance measuring tools (like Valgrind), if
  appropriate
\item Libraries you will likely be using?
\item Tools you will likely be using?
\end{itemize}

\section{Coding Standard}

\section{Project Scheduling}

\wss{How will the project be scheduled?}

\end{document}
